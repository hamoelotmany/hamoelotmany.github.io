\documentclass[twoside,11pt,a4paper,]{book}
\usepackage[T1]{fontenc}
\usepackage[french]{babel}
\usepackage{palatino}
\usepackage{mathpazo}
\usepackage{hyperref}
\usepackage{xspace}  
\usepackage{lipsum}
\usepackage{cite}
\usepackage{lettrine} 
\usepackage{eso-pic} 
\usepackage{titlesec,titletoc}
\usepackage{pifont}
\usepackage{eurosym}
\usepackage[french]{minitoc}
\usepackage{setspace}

\usepackage{fancyhdr}
 \setlength{\headheight}{14.2pt}
\renewcommand{\chaptermark}[1]{
\markboth{\bsc{\chaptername~\thechapter{} :} #1}{}} 
\renewcommand{\sectionmark}[1]{\markright{\bsc{\thesection{}} #1}}
 
 \usepackage[centertags,sumlimits,intlimits]{amsmath} 
 \usepackage{amssymb,amsfonts,amsthm,amstext,amsopn,latexsym}
 \usepackage{mathrsfs} 
 \usepackage{pifont}
 \usepackage{wasysym}

\newtheorem{theoreme}{Th�or�me}[section]
\newtheorem{proposition}{Proposition}[section]
\newtheorem{lemme} {Lemme}[section]
\newtheorem{corollaire}{Corollaire}[section]
\newtheorem{exemple}{Exemple}[section]
\newtheorem{remarque}{Remarque}[section]
\newtheorem{definition}{D�finition}[section]

 \usepackage{floatflt}
 \usepackage{float}
 \usepackage{graphicx}
  \usepackage{tabularx}
 \usepackage{subfigure}
 \usepackage{epsfig}
  \usepackage{multicol}
  \usepackage{multirow}



 %%============================================================================
%=================              PAGE          ===============================
%%============================================================================

 \usepackage{geometry}
 \geometry{verbose,a4paper,
   includeheadfoot,headheight=16pt,
   textheight=235mm,textwidth=150mm,
   vmarginratio=20:25,hmarginratio=20:25}
 

  %% ============================================================================
  %% =================            CAPTION         ===============================
  %% ============================================================================

  \usepackage{caption}
  \captionsetup{figurename=Fig., tablename=Tab.,
    labelformat=simple,font=footnotesize,format=plain,
    justification= justified,
    labelsep=endash,labelfont=bf}

%% ============================================================================
 %% =================         d�finir deux ligne horizontales     ===============================
 %% ============================================================================ 

\newcommand\HRule{\noindent\rule{\linewidth}{1.5pt}}
\newcommand\HHRule{\noindent\rule{\linewidth}{1.5pt}}
  %% ============================================================================
  %% =================            Perso         =================================
  %% ============================================================================

  \newcommand{\couleur}{\textcolor{red}}
 \def\boldmath{\mathversion{bold}}
\def\bm#1{\mathchoice
                                        {\mbox{\boldmath$\displaystyle#1$}}%
          {\mbox{\boldmath$#1$}}%
          {\mbox{\boldmath$\scriptstyle#1$}}%
          {\mbox{\boldmath$\scriptscriptstyle#1$}}}
           
     % vitesse de d�formation gamma
   \newcommand{\gp}{\overset{.}{\gamma}}
   % vitesse de d�formation epsilon
   \newcommand{\ep}{\overset{.}{\varepsilon}}
   
   \newcommand{\DDt}[1]{{\frac{\mathcal{D}{#1}}{\mathcal{D}{t}}}}
   % d�riv�e corotationnelle
   \newcommand{\dcr}[1]{{\stackrel{\circ}{\boldsymbol{#1}}}}
   % d�riv�e convect�e contravariante
   \newcommand{\dct}[1]{{\stackrel{\triangledown}{\boldsymbol{#1}}}}
   % d�riv�e convect�e covariante
   \newcommand{\dco}[1]{{\stackrel{\vartriangle}{\boldsymbol{#1}}}}
   % d�riv�e de Gordon-Schowalter
   \newcommand{\dgo}[1]{{\stackrel{\Box}{\boldsymbol{#1}}}}
             
  \newcommand{\vct}[1]{\boldsymbol{#1}}
  \newcommand{\tns}[1]{\boldsymbol{#1}}
    
  \newcommand{\du}[1]{\mathrm{d}{#1}}
  \newcommand{\dvt}[1]{\mathrm{d}{\boldsymbol{#1}}}
    
  \newcommand{\dfn}[2]{\partial^{#1}{#2}}
  \newcommand{\dfd}[2]{\partial{{#1}}^{#2}}
    
  \newcommand{\dfg}[2]{\partial{#1}^{#2}}
  \newcommand{\ddt}[1]{\frac{\mathrm{d}{#1}}{\mathrm{d}t}}
  \newcommand{\dds}[2]{\frac{\mathrm{d}{#1}}{\mathrm{d}{#2}}}
  \newcommand{\dtp}[1]{\frac{\partial{#1}}{\partial{t}}}
  \newcommand{\dsp}[2]{\frac{\partial{#1}}{\partial{#2}}}


 %%============================================================================
 %% =================                 Annexes          ========================================
 %% ============================================================================
 
\usepackage[title,titletoc]{appendix}
\renewcommand{\appendixtocname}{Annexe}

 %% ============================================================================
 %% =================         LE DOCUMENT          ========================================
 %% ============================================================================ 

\setcounter{minitocdepth}{2}
\setlength{\mtcindent}{24pt}
\renewcommand{\mtcfont}{\small\rm}
\renewcommand{\mtcSfont}{\small\bf}

\begin{document}

%% ==============================================================================
%% Fichier : LaTeX
%% Auteur : Hammou El-Otmany
%% E-Mail  : hammou.elotmany@univ-pau.fr
%% Exposé : PAGE DE GARDE
%% Date     : 07/7/2015
%% ==============================================================================

%TITRE : Approximation par la m\'ethode NXFEM des probl\`emes d'interface et d'interphase dans la m\'ecanique des fluides

\makeatletter
\def\@ecole{\'ecole}
\newcommand{\ecole}[1]{
\def\@ecole{#1}
}
 
\def\@specialite{Sp\'ecialit\'e}
\newcommand{\specialite}[1]{
\def\@specialite{#1}
}
 
\def\@ED{\'{E}cole Doctorale}
\newcommand{\ED}[1]{
\def\@ED{#1}
}
 
\def\@doctorat{Doctorat}
\newcommand{\doctorat}[1]{
\def\@doctorat{#1}
}
 
\def\@adresse{Adresse}
\newcommand{\adresse}[1]{
\def\@adresse{#1}
}
 
\def\@directrice{directrice}
\newcommand{\directrice}[1]{
\def\@directrice{#1}
}
 
\def\@directeur{directeur}
\newcommand{\directeur}[1]{
\def\@directeur{#1}
}
\def\@jurya{}{}{}{}
\newcommand{\jurya}[4]{
\def\@jurya{#1 & #2 & #3 & #4\\}
}
\def\@juryb{}{}{}{}
\newcommand{\juryb}[4]{
\def\@juryb{#1 & #2 & #3 & #4\\}
}
\def\@juryc{}{}{}{}
\newcommand{\juryc}[4]{
\def\@juryc{#1 & #2 & #3 & #4\\}
}
\def\@juryd{}{}{}{}
\newcommand{\juryd}[4]{
\def\@juryd{#1 & #2 & #3 & #4\\}
}
\def\@jurye{}{}{}{}
\newcommand{\jurye}[4]{
\def\@jurye{#1 & #2 & #3 & #4\\}
}
\def\@juryf{}{}{}{}
\newcommand{\juryf}[4]{
\def\@juryf{#1 & #2 & #3 & #4 \\}
}
\def\@juryg{}{}{}{}
\newcommand{\juryg}[4]{
\def\@juryg{#1 & #2 & #3 & #4\\}
}
\def\@juryh{}{}{}{}
\newcommand{\juryh}[4]{
\def\@juryh{#1 & #2 & #3\\}
}
\def\@juryi{}{}{}{}
\newcommand{\juryi}[4]{
\def\@juryi{#1 & #2 & #3\\}
}
\def\@juryr{}{}{}{}
\newcommand{\juryr}[4]{
\def\@juryr{#1 & #2 & #3\\}
}
\def\@juryv{}{}{}{}
\newcommand{\juryv}[4]{
\def\@juryv{#1 & #2 & #3\\}
}

 
\makeatletter
\newcommand{\Makepagedegarde}{
       \newgeometry{top=2.5cm, bottom=1cm, left=2cm, right=1cm}
      \begin{titlepage}
            \begin{center}
            \hspace*{.2cm}\includegraphics[width=0.3\textwidth]{Pagarde/uppa.png}
           \hfill
           \hspace*{-.2cm}\includegraphics[width=0.1\textwidth]{Pagarde/cnrs.jpg}\\
          \vspace{1cm}
          {\Large \@ED}\\
          \vspace{1cm}
         {\huge
               {\bfseries \@doctorat}  }\\
          \vspace{1cm}
         { pr\'esent\'ee pour obtenir le grade de }\\
         \vspace{1cm}
         {\huge\bfseries DOCTEUR}\\
         \vspace{1cm}
         {\textsc{\huge\bfseries De l'\@ecole}}\\
         \vspace{0.5cm}
         {\Large{\bfseries \textit{ Sp\'ecialit\'e : \@specialite }}}\\
         \vspace{1cm}
         { par }\\
          \vspace{.6cm}
           {\Large {\bfseries \@author}} \\
           \HRule  \\ \vspace*{.5cm} 
            {\LARGE \bfseries{\@title}} \\
            \HRule \\
             \vspace{.5cm}
             {soutenue publiquement le 9 novembre 2015} \\ %\@date
             \end{center}
\textbf{\underline{Apr\`es avis de : }}\\
\vspace*{.55cm} 
       \begin{tabular}{llll}
               \@juryr
               \@juryv
        \end{tabular}\\
\textbf{\underline{Devant la commission d'examen compos\'ee de : }}\\
\vspace*{.55cm} 
        \begin{tabular}{llll}
               \@jurya
               \@juryb
               \@juryc
               \@juryd
               \@jurye
               \@juryf
               \@juryg
               \@juryh
               \@juryi
         \end{tabular}
          \vfill
          \centering
         \@adresse
    \end{titlepage}
 
 
 
 
 \restoregeometry
} 
 
\author{Hammou \textsc{EL-OTMANY}}
\title{APPROXIMATION PAR LA M\'ETHODE NXFEM DES PROBL\`EMES D'INTERFACE ET D'INTERPHASE EN M\'ECANIQUE DES FLUIDES}
\ED{\'ECOLE DOCTORALE DES SCIENCES ET LEURS APPLICATIONS-ED \no 211}
\doctorat{TH\`ESE}
\specialite{Math\'ematiques Appliqu\'ees}
\date{\today}
\jurya{M. }{BLOUZA Adel }{Ma\^itre de Conf\'erences HDR, Universit\'e de Rouen}{Rapporteur}
\juryb{Mme}{CAPATINA Daniela}{Ma\^itre de Conf\'erences HDR, Universit\'e de Pau}{Directrice}
\juryc{M. }{EYMARD Robert }{Professeur, Universit\'e Paris-Est Marne-la-Vall\'ee}{Pr\'esident}
\juryd{M. }{GRAEBLING Didier}{Professeur, Universit\'e de Pau}{Directeur}
\jurye{M. }{HILD Patrick }{Professeur, Universit\'e Paul Sabatier - Toulouse 3}{Rapporteur}
\juryf{M.  }{LUCE Robert}{Ma\^itre de Conf\'erences HDR, Universit\'e de Pau}{Examinateur}
\juryr{M. \quad}{BLOUZA Adel \quad\quad}{Ma\^itre de Conf\'erences HDR, Universit\'e de Rouen}{Rapporteur}
\juryv{M. \quad}{HILD Patrick }{Professeur, Universit\'e Paul Sabatier - Toulouse 3}{Rapporteur}
\ecole{Universit\'e de Pau et des Pays de l'Adour}
\adresse{
\textbf{Laboratoire de Math\'ematiques et de leurs Applications de Pau (LMAP)}\\
	UMR CNRS 5142, Avenue de l'Universit\'e, 64012 Pau Cedex, France
} 
\Makepagedegarde

\thispagestyle{empty} 
\vspace*{-8cm}
       \chapter*{}
  
\input{Merci/Remerci}


\tableofcontents

\cleardoublepage
\chapter*{Notations g\'en\'erales}
\markboth{Notations g\'en\'erales}{}
\mtcaddchapter[Notations g\'en\'erales] 
\label{cha:notations}

\noindent  

\input{Texte/Introduction}
%% ============================================================================ 


   \chapter*{}

   \thispagestyle{empty}

   \addstarredchapter{Partie I. Probl�me d'interface : extension de la
     m�thode NXFEM aux �l�ments finis non-conformes}

   \begin{center}
     \linespread{3}
     \begin{center}
       \begin{minipage}[c]{\textwidth}
         \setlength{\baselineskip}{5ex} \centering
         { \Huge \sc  Partie I}\\[.6cm]
         { \huge \sc  Probl�me d'interface :
           extension de la m�thode NXFEM aux �l�ments finis non-conformes}\\[0.4cm]

       \end{minipage}
     \end{center}
   \end{center}


   \newpage

   \thispagestyle{empty} 	 
\input{Texte/Chap1}         
\input{Texte/Chap2}        
\input{Texte/Chap3}        

%% ============================================================================ 



    \chapter*{}

    \thispagestyle{empty}

    \addstarredchapter{Partie II. Probl�me d'interphase : mod�lisation
      asymptotique et approximation par NXFEM}

    \begin{center}
      \linespread{3}
      \begin{center}
        \begin{minipage}[c]{\textwidth}
          \setlength{\baselineskip}{5ex}
          \centering
          { \Huge \sc  Partie II}\\[.6cm]
          { \huge \sc  Probl�me d'interphase 
            : mod�lisation asymptotique et approximation par NXFEM}\\[0.4cm]
        \end{minipage}
      \end{center}
    \end{center}

  

    \newpage

    \thispagestyle{empty}          
\input{Texte/Chap4}     
\input{Texte/Chap5}     
\input{Texte/Chap6}     

%% ============================================================================ 


     \chapter*{}

     \thispagestyle{empty}

     \addstarredchapter{Partie III. Mod�lisation d'une membrane par un
       fluide non-newtonien}

     \null

     \begin{center}
       \linespread{3}
       \begin{center}
         \begin{minipage}[c]{\textwidth}
           \setlength{\baselineskip}{5ex} \centering
           { \Huge \sc  Partie III}\\[.6cm]
           { \huge \sc Mod�lisation d'une membrane par un fluide
             non-newtonien}\\[0.4cm]
         \end{minipage}
       \end{center}
     \end{center}



     \newpage

     \thispagestyle{empty}
     
           
     \chapter{Globules rouges et mod�les
       rh�ologiques}\label{Chap07}
        
            Dans ce chapitre, nous nous int�ressons aux relations entre le comportement m�canique des globules rouges et l'�coulement sanguin. Cela revient � appr�hender l'h�modynamique et la rh�ologie du sang � travers le comportement m�canique des globules rouges. Pour cela, nous avons choisi le mod�le rh�ologique visco�lastique non-newtonien et non-lin�aire de Giesekus pour mod�liser le comportement de la membrane cellulaire. 
     \hfill
     \begin{spacing}{0.8}
       \minitoc    
     \end{spacing}     
\input{Texte/Chap8}     
\input{Texte/Chap9}     

%% ============================================================================ 


\addcontentsline{toc}{chapter}{Annexes}
\appendix
\input{Texte/Annexe}      

%% ============================================================================ 

\addcontentsline{toc}{chapter}{Bibliographie}
\renewcommand{\bibname}{Bibliographie}
\bibliographystyle{plain}
\bibliography{Biblio/Reference}

\clearpage
\newpage
\strut  \mbox{}  \null
\newpage      

\end{document}

